\chapter{Аналитический раздел}%
\label{cha:analiticheskii_razdel}

В данном разделе расмматриваются имеющиейся на рынке решения, производится анализ существующих моделей БД и СУБД.

\section{Постановка задачи}%
\label{sec:postanovka_zadachi}

В соответствии с техническим заданием ну курсовую работу требуется разработать клиент-серверное web-приложение, через которое осуществляется доступ чтения, добавления и обновления к находящимся в базе данных данным пользователей, их портфолио, профессиям, записям и комментариям. Это приложение должно обладать следующей функциональностью:
\begin{itemize}
    \item создание, выборка, изменение и удаление записей с данными пользователя (имя пользователя, пароль, имя, фамилия и т.д.);
    \item создание, выборка, изменение и удаление записей с данными о професиях;
    \item создание, выборка, изменение и удаление записей с данными о портфолио пользователй;
    \item создание, выборка, изменение и удаление записей с данными о записях пользователей;
    \item создание, выборка, изменение и удаление записей с данными о комментариях пользователей;
    \item поиск пользователей по их имени пользователя.
\end{itemize}

\section{Анализ существующих решений}%
\label{sec:analiz_sushchestvuiushchikh_reshenii}

На сегодняшний день на рынке имеются различные продукты, решающие поставленную цель в разной степени. Для их рассмотрения и анализа введём следующие критерии:
\begin{enumerate}[label={\arabic*)}]
    \item возможность создания и публикации резюме/портфолио (далее просто портфолио);
    \item возможность ведения блога/написания статей в контексте определённого портфолио;
    \item возможность комментирования записей;
    \item возможность поиска соискателей;
    \item доступность сервиса (характеристика легкости в получении пользователями доступа к функциональности).
\end{enumerate}

\subsection{LinkedIn}%
\label{sub:linkedin}

\textbf{LinkedIn}~--- это социальная сеть для поиска и установления деловых контактов. Она позволяет публиковать профессиональные резюме, осуществлять поиск работы, рекомендовать людей и быть рекомендованным, публиковать вакансии и создавать группы по интересам.

Для доступа к LinkedIn необходимо пройти регистрацию, кроме того данный сервис заблокирован на территории Российской Федерации.

\subsection{Medium}%
\label{sub:medium}

\textbf{Medium}~--- это платформа для социальной журналистики. Основными её возможностями является создания и публикация статей на различные темы, комментирование этих статей и различное их продвижение. Просмотр статей на этом ресурсе регистрацией не ограничивается.

\subsection{Хабр}%
\label{sub:khabr}

\textbf{Хабр}~--- это веб-сайт в системе тематических коллективных блогов. Хабр предоставляет доступ для создания, публикации новостей, аналитических статей, мыслей относительно различных технологиях и инфоповодами, связанными с ними.

Доступ на чтение открыт для всех без исключения, но возможность публикации ограничена не только регистрацией, но и рядом дополнительных требований, служащих гарантом качества публикуемых статей.

\subsection{Вывод}%
\label{sub:vyvod_}

По итогу анализа представленных решений, имеем следующую таблицу:

\begin{table}[H]
    \stepcounter{tableqty}
    \label{tab:products}
    \caption{Анализ существующих решений}
    \centering
    \begin{tabular}{l*{5}{>{\centering\arraybackslash}m{1cm}}}
 Номер критерия & 1 & 2 & 3 & 4 & 5 \\
 \midrule
 LinkedIn & + & -- & -- & + & -- \\
 Medium   & -- & + & + & + & + \\
 Хабр     & -- & + & + & + & +/-- \\
\end{tabular}
\end{table}

Отсюда можно сделать вывод, что на рынке отсутствует продукт в полной мере достигающей поставленной цели.

\section{Анализ и выбор модели БД}%
\label{sec:analiz_i_vybor_modeli_bd}

Модель данных~--- это совокупность структур данных и операций их обработки. С помощью модели данных могут быть представлены объекты предметной области и взаимосвязи между ними~\cite{db}. Модель данных представляет сочетание следующих компонентов:
\begin{itemize}
    \item структурная часть (набор правил, по которым может быть построена база данных);
    \item управляющая часть (определяет типы дополнительных операций над данными).
\end{itemize}

Различают следующие основные четыре модели данных~\cite{db}:
\begin{itemize}
    \item иерархическая;
    \item сетевая;
    \item объектно-ориентированная;
    \item реляционная.
\end{itemize}

Рассмотрим эти модели подробнее.

\subsection{Иерархическая модель}%
\label{sub:ierarkhicheskaia_model_}

Иерархические базы данных~--- это базы данных с древовидной структурой с дугами-связями и узлами-элементами данных. Данная структура предполагает жесткое подчинение между данными в системами. Основной недостаток данной модели~--- подобная структура не удовлетворяет требованиям многих реальных задач.

\subsection{Сетевая модель}%
\label{sub:setevaia_model_}

В сетевых базах данных наряду с вертикальными связями допустимы и горизонтальные связи. Однако в данной модели имеют место многие недостатки иерархической модели и главный из них~--- необходимость четко определять на физическом уровне связи данных и столь же четко следовать этой структуре связей при запросах к базе.

\subsection{Объектно-ориентированная модель}%
\label{sub:ob_ektno_orientirovannaia_model_}

Новые области применения средств вычислительной техники, такие автоматизированное проектирование, потребовали от баз данных способности хранить и обрабатывать новые объекты: текст, аудио информацию, видео информацию и прочее. Основные трудности объектно-ориентированного моделирования данных проистекают из того, что такого развитого математического аппарата, на который могла бы опираться общая объектно-ориентированная модель данных, не существует.

\subsection{Реляционная}%
\label{sub:reliatsionnaia}

Реляционная модель появилась вследствие стремления сделать базу данных максимально гибкой. Данная модель представила простой и эффективный механизм поддержания связей данных. Все данные в модели представляются только в виде таблиц.

\subsection{Выбор}%
\label{sub:vybor_db}

Реляционная модель данных имеет ряд преимуществ в сравнении с остальными рассмотренными моделями. Например, гибкость и целостность данных. Именно поэтому для реализации данной курсовой работы была выбрана реляционная модель.

\section{Анализ и выбор СУБД}%
\label{sec:analiz_i_vybor_subd}

Система управления базами данных~--- это комплекс программно-языковых средств, позволяющих создавать базы данных и управлять ими. Среди самых популярных реляционных СУБД можно выделить~\cite{subd}:
\begin{itemize}
    \item MySQL;
    \item MS SQL Server;
    \item SQLite
    \item PostgreSQL;
\end{itemize}

Для выбора СУБД определим следующие критерии:
\begin{enumerate}
    \item свободное распространение;
    \item полная поддержка SQL;
    \item наличие ORM;
    \item лёгкость развёртывания;
\end{enumerate}

\subsection{MySQL}%
\label{sub:mysql}

\textbf{MySQL}~--- это СУБД с открытым исходным кодом. Является одной из самой популярной реляционной СУБД. Основными достоинствами данной системы управления базами данных являются:
\begin{itemize}
    \item простота при работе с ней;
    \item безопасность;
    \item масштабируем ость;
    \item поддержка большей части функционала SQL;
    \item скорость.
\end{itemize}
При этом у MySQL есть существенные недостатки:
\begin{itemize}
    \item фрагментарное использование SQL;
    \item подверженность DDos-атакам;
    \item Платная техническая поддержка.
\end{itemize}

\subsection{MS SQL Server}%
\label{sub:ms_sql_server}

\textbf{MS SQL Server}~--- это СУБД, разрабатываемая компанией Microsoft. Достоинства:
\begin{itemize}
    \item высокая масштабируемость;
    \item синхронизация с другими продуктами Microsoft;
    \item хорошая защита данных.
\end{itemize}

Недостатки:
\begin{itemize}
    \item платная схема распространения;
    \item повышенное потребление ресурсов;
\end{itemize}

\subsection{SQLite}%
\label{sub:sqlite}
\textbf{SQLite}~--- компактная встраиваемая СУБД. Исходный код библиотеки передан в общественное достояние.

Достоинства:
\begin{itemize}
    \item файловая структура~--- вся база данных состоит из одного файла;
    \item масштабируемость.
\end{itemize}

 Недостатки:
 \begin{itemize}
     \item фрагментарное использование;
     \item отсутствие системы пользователей;
 \end{itemize}

 \subsection{PostgreSQL}%
 \label{sub:postgresql}
 
\textbf{PostgreSQL}~--- самая продвинутая современная СУБД с открытым исходным кодом. Данное ПО обладаем следующими преимуществами:
\begin{itemize}
    \item возможность использовать множество дополнений помимо мощного
встроенного SQL;
    \item существование большого сообщества пользователей, благодаря которым
можно найти ответ на любой вопрос и решить любую проблему;
    \item поддержка множеством организаций как одного из самых перспективных
    \item поддержка json, csv.
\end{itemize}

Недостатки:
\begin{itemize}
    \item повышенный расход ресурсов;
\end{itemize}

\subsection{Выбор}%
\label{sub:vybor_dbms}

\begin{table}[H]
    \stepcounter{tableqty}
    \label{tab:dbms}
    \caption{Анализ СУБД}
    \centering
    \begin{tabular}{l*{4}{>{\centering\arraybackslash}m{1cm}}}
 Номер критерия & 1 & 2 & 3 & 4 \\
 \midrule
 MySQL           & + & +/-- & + & +/-- \\
 MS SQL Server   & -- & +   & + & +/-- \\
 SQLite          & + & +/-- & + & +   \\
 PostgreSQL      & + & +   & + & +/-- \\
\end{tabular}
\end{table}

Таким образом, выбор остается сделать из двух СУБД: SQLite и PostgreSQL. Так как в перспективе полная поддержка SQL важнее лёгкости развёртывания, в контексте данной курсовой работы выбор был сделан в пользу PostgreSQL.

\section{Выводы}%
\label{sec:vyvody}

В данном разделе была произведена постановка задачи, проведен анализ существующих решений, выбрана в качестве модели данных реляционная модель данных, и PostgreSQL~--- в качестве СУБД.
