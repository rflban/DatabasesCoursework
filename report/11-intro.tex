\chapter*{\hfill{}ВВЕДЕНИЕ\hfill{}}%
\label{cha:vvedenie}
\addcontentsline{toc}{chapter}{ВВЕДЕНИЕ}

Конкуренция на рынке труда среди соискателей приводит к необходимости владения претендентами навыком представления своих лучших черт, соответствующих целевой профессии. Очевидно, что данное умение человека испытывается на собеседовании, а так же на предшествующем ему этапе создания резюме.

Проблема написания резюме нетривиальна и обладает значительным количеством факторов, которые критически важно учесть. Одним из таких факторов является лаконичность. Чересчур объёмный и сложный текст будет отталкивать работодателя. Таким образом, резюме представляет собой лишённый деталей набор фактов о потенциальном работнике. Значит ли это, что детали как таковые не нужны для представления претендента перед его возможным нанимателем? Конечно, нет, ведь именно детали дают полную картину человека, выражают степень его компетентности, увлеченности работой и прочих характеристик.

Как было установлено, резюме должно быть лаконичным. Но тогда возникает вопрос: как соискатель может рассказать о себе больше? Люди по-разному решают эту проблему: некоторые ведут посвященный своей специальности блог, кто-то пишет статьи, а иные~--- создают сайты-визитки. Для достижения максимальной презентации работником его навыков можно объединить все эти способы и резюме в одном месте, а так же предоставить удобные инструменты для управления полученной системы.

Таким образом, \textbf{целью} данной курсовой работы создание клиент-серверного web-приложения, которое позволяет пользователям регистрироваться, авторизироваться, и создавать портфолио для различных профессий.

Для достижения поставленной цели необходимо решить следующие задачи:
\begin{itemize}
    \item проанализировать имеющиеся на рынке решения;
    \item провести сравнительный анализ существующих моделей БД и СУБД;
    \item спроектировать базу данных, обеспечивающую структурное хранение данных;
    \item с использованием выбранной СУБД реализовать спроектированную базу данных;
    \item реализовать клиент-серверное web-приложение для взаимодействия с имеющейся базой данных;
    \item проверить работоспособность разработанного приложения.
\end{itemize}
